\documentclass{article}
\usepackage[utf8]{inputenc}
\usepackage{polski}
\usepackage{fullpage}
\usepackage{amsmath}
\title{Zadanie 3 lista 6}
\author{Kajetan Bilski 244942}

\begin{document}
	\maketitle

Mamy gramatykę:\\
$S \to E$\\
$E \to E \text{ or } T|T$\\
$T \to T \text{ and } F|F$\\
$F \to \text{not } F|(E)|\text{true}|\text{false}$\\
Eliminacja lewostronnej rekurencji:\\
$S \to E$\\
$E \to TG$\\
$G \to \text{or }TG|\varepsilon$\\
$T \to FV$\\
$V \to \text{and }FV|\varepsilon$\\
$F \to \text{not } F|(E)|\text{true}|\text{false}$\\
W powstałej gramatyce nie ma powtarzających się prefiksów prawych stron wyprowadzeń, więc nie ma jak ani po co wykonać lewostronnej faktoryzacji.\\
$FIRST(S) = FIRST(E)$\\
$FIRST(E) = FIRST(T)$\\
$FIRST(G) = \{ \text{or},\varepsilon\} $\\
$FIRST(T) = FIRST(F)$\\
$FIRST(V) = \{ \text{and},\varepsilon\} $\\
$FIRST(F) = \{ \text{not},( ,\text{true},\text{false}\} $\\
$FOLLOW(S) = \{ \$ \} $\\
$FOLLOW(E) = \{ \$ ,)\} $\\
$FOLLOW(G) = FOLLOW(E)$\\
$FOLLOW(T) = \{ \text{or},\$ ,) \} $\\
$FOLLOW(V) = FOLLOW(T)$\\
$FOLLOW(F) = \{ \text{and},\text{or},\$ ,) \} $\\
Tablica przejść $LL(1)$:\\
\begin{center}
	\begin{tabular} {c | c | c | c | c | c | c | c | c}
		& or & and & not & ( & ) & true & false & \$ \\ \hline
		S & & & E & E & & E & E & \\ \hline
		E & & & TG & TG & & TG & TG & \\ \hline
		G & or TG & & & & $\varepsilon$ & & & $\varepsilon$ \\ \hline
		T & & & FV & FV & & FV & FV & \\ \hline
		V & $\varepsilon$ & and FV & & & $\varepsilon$ & & & $\varepsilon$ \\ \hline
		F & & & not F & (E) & & true & false & \\
	\end{tabular}
\end{center}
Da się stworzyć dla tej gramatyki tablicę przejść parsera typu $LL(1)$, co znaczy że jest ona typu $LL(1)$.
\end{document}