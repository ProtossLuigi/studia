\documentclass{article}
\usepackage[utf8]{inputenc}
\usepackage{polski}
\title{Sprawozdanie}
\author{Kajetan Bilski 244942}

\begin{document}
	\maketitle
	\pagenumbering{arabic}

\section{Zadanie 1.}
W tym zadaniu trzeba napisać funkcję znajdującą miejsce zerowe podanej funkcji metodą bisekcji według specyfikacji podanej w poleceniu zadania.\\
Kod funkcji do zadań 1 - 3 jest w pliku zad123.jl.
Funkcja najpierw sprawdza znaki wartości funkcji na końcach danego przedziału. Jeśli $sign(f(a_{0})) = sign(f(b_{0}))$ to zwraca kod błędu 1. W przeciwnym wypdaku liczy $x_{0}=\frac{b_{0}-a_{0}}{2}+a_{0}$ i wchodzi do pętli $while$.
\section{Zadanie 2.}

\section{Zadanie 3.}

\section{Zadanie 4.}

\section{Zadanie 5.}

\section{Zadanie 6.}
\end{document}